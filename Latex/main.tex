\documentclass[a4paper,14pt]{extreport}
\usepackage[utf8]{inputenc}
\renewcommand{\baselinestretch}{1.5}
\renewcommand\labelitemi{--}

\makeatletter
% запрещаем переносы в названиях секций
\renewcommand{\section}{\@startsection{section}{1}{0pt}%
	{-3.5ex plus -1ex minus -.2ex}%
	{2.3ex plus .2ex}%
	{\centering\hyphenpenalty=10000\normalfont\Large\bfseries}}

% меняем заголовок для команды \chapter и запрещаем переносы слов
\usepackage {titlesec}
\titleformat{\chapter}{\thispagestyle{myheadings}\centering\hyphenpenalty=10000\normalfont\huge\bfseries}{
	\thechapter. }{0pt}{\huge}
\makeatother

\setlength{\parindent}{10mm}

% Переопределение команды подсекции
\makeatletter
\renewcommand{\subsection}{\@startsection{subsection}{2}%
	{\parindent}{3.25ex plus 1ex minus .2ex}%
	{1.5ex plus .2ex}{\bfseries}}
\makeatother


\newlength{\curtextsize}
\newlength{\bigtextsize}
\setlength{\bigtextsize}{11.9pt}


\makeatletter
\show\f@size
\setlength{\curtextsize}{\f@size pt}
\makeatother

\makeatletter
\def\@seccntformat#1{%
	\expandafter\ifx\csname c@#1\endcsname\c@section\else
	\csname the#1\endcsname\quad
	\fi}
\makeatother

%%%%%%%
\usepackage[english, russian]{babel}
\usepackage{pgfplots}
\usepackage{amsmath}
\usepackage{tikz}	
\usepackage{amssymb}
\usepackage{cite}
\usepackage {indentfirst}
\usepackage{geometry}
\geometry{verbose,a4paper,tmargin=2cm,bmargin=2cm,lmargin=3cm,rmargin=2.0cm}
\usepackage{hyperref}

\title{Литературный обзор}
\author{Антон Максимов, 527 группа}
\date{}
\begin{document}
	\setcounter{tocdepth}{2}
	\renewcommand{\contentsname}{Содержание}
	\tableofcontents
	\newpage
	%\maketitle
	\bibliographystyle{unsrt}
	
	\chapter{Введение}
	\textit{(пока что оформление не по ГОСТу, не ставится на TexStudio, но надежда есть)}	
	\section{Цели и актуальность работы}
	Целью дипломной работы является разработка гибкого протокола поиска и оценки \textbf{off-target} эффектов разного типа у выбранного белка и его лигандов. Исходный белок не имеет экспериментально полученной пространственной структуры. Основная цель протокола -- работа с рецепторами, сопряженными с G-белком.
	
	Насколько нам (мне) известно, в мире не существует публично доступных протоколов, позволяющих по последовательности мишени произвести основанный на структурах лиганда, мишени и связывающего кармана поиск всевозможных \textbf{off-target} эффектов как со стороны лигандов, так и со стороны мишени.
	
	Возникновение такого протокола позволит оценивать перспективность разработки или \textbf{respurposing} лекарств и необходимость расшифровки пространственной структуры мишени экспериментальными методами.
	\section{Список используемых сокращений}
	\noindent GPCR -- рецепторы, сопряженные с G-белком (англ. G-protein-coupled receptors)\\
	ТМ -- трансмембранный участок\\
	МГ -- моделирование по гомологии\\
	МД -- молекулярная динамика
	
	\chapter{Обзор литературы}
	\section{Рецепторы, сопряженные с G-белком}
	Семейство GPCR состоит из около 800 многофункциональных белков-рецепторов \cite{Fredriksson1256}, регулирующих разнообразные внутриклеточные сигнальные каскады в ответ на гормоны, нейротрансмиттеры, ионы, фотоны, одоранты и другие стимулы\cite{Hilger2018}. Поэтому они играют важнейшую роль в физиологии и разработке лекарств, представляя собой привлекательную мишень для лекарственных средств. Около трети всех лекарств, одобренных Управлением по санитарному надзору за качеством пищевых продуктов и медикаментов США, действуют именно на мишени из этого класса, хотя это всего чуть более ста различных рецепторов, то есть порядка десятой части всего семейства\cite{Hauser2017}.
	
	Белки, принадлежащие семейству GPCR, состоят из семи трансмембранных спиралей, связанных тремя внутриклеточными и внеклеточными петлями. Внеклеточная часть, с которой связывается лиганд, также включает в себя N-конец. Внутриклеточная часть содержит кроме петель также восьмую спираль и C-конец и взаимодействует с G белками, аррестинами и другими \textbf{downstream effectors}	\cite{Katritch2012Diversity}. 
	
	Трансмембранная часть является наиболее консервативной в структуре белков семейства, что не мешает рецепторам из различных подсемейств обеспечивать крайнее разнообразие в форме, размере и электростатических свойствах связывающихся с ними лигандов за счет вариаций в структуре связывающих карманов. 	
	Их гидрофобность и закрытость с внеклеточной стороны связаны с функциями рецептора \cite{Katritch2012Diversity}.
	
	Несмотря на колоссальный прогресс в кристаллографии GPCR, пространственная структура известна только у небольшой доли рецепторов. Важная роль компьютерного моделирования состоит в раскрытии структур остальных рецепторов и их комплексов \cite{KUFAREVA20141120} и последующем рациональном создании лекарств.
	\section{Сайты связывания и специфичность}
	Хотя природные лиганды внутри семейства GPCR очень разнообразны, белки одного подтипа имеют практически одинаковые конформации активных сайтов, что позволяет моделировать их компьютерными методами с высокой точностью
	\cite{Katritch2012}. Некоторые подсемейства рецепторов взаимодействуют с одним и тем же \textbf{endogenous} лигандом, и в этом случае отсутствие больших различий в ортостерических связывающих карманах, где и происходит взаимодействие \textbf{endogenous} лиганда и рецептора, представляет вызов для поиска селективных лигандов и явялется одной из главных проблем в разработке безопасных и эффективных лекарств, действующих на GPCR \cite{KATRITCH2011108}.
	
	Исторически дизайн лекарств был направлен на создание лигандов, подобных \textbf{endogenous}, которые искусственно активировали сигнальные пути. Другой классический подход состоит в создании антагонистов~-- веществ, способных конкурировать с природным лигандом, не активируя при этом работу рецептора. 
	
	В последнее время произошел взрывной рост количества новых методов использования GPCR в качестве мишени. Например, многие недавно разработанные лиганды действуют в активном сайте, топологически отделенном от ортостерического. Такие сайты и лиганды называются <<аллостерическими>>, причем лиганды могут как усиливать работу рецептора, так и ослаблять ее \cite{SHONBERG20153880}. Возможна даже комбинация фармакофоров ортостерических и аллостерических лигандов (\textbf{bitopic ligands}) в одном лиганде, который будут теоретически иметь лучшую аффинность и селективность за счет большего количества связей с рецептором\cite{Wootten2013}.
	
	Простейший механизм активации GPCR включает в себя два состояния, между которыми балансирует рецептор: активное, в котором происходит передача сигнала внутрь клетки, и неактивное. Агонисты смещают это равновесие в сторону активной конформации, обратные агонисты - в сторону неактивного. Эффективность агониста определяется преобладанием \textit{аффинности к активному состоянию над неактивным(звучит коряво)}. К настоящему моменту стало очевидно, что различные лиганды, воздействуя на один и тот же рецептор, могут стабилизировать его в различных конформациях, так что активными остается только часть всех возможных сигнальных путей, в который вовлечен этот рецептор. Такой процесс называется \textbf{biased} агонизм \cite{Lane2017}.
	%	\cite{Vasile2018}
	\section{Пространство лигандов для GPCR}
	\textit{(не очень понятно, что нужно писать здесь. Перечисление типов найденных для GPCR лигандов, их особенности? (но у всех разные вроде))}
	\section{Полифармакология}
	При разработке лекарств важно добиться селективности, избавившись от побочных действий. Именно эта парадигма <<одно лекарство -- одна мишень>>, так называемая таргетированная терапия, до недавнего времени широко использовалась в фармакологии. С другой стороны, в последнее время стала осознаваться важность полифармакологии, которая означает множественное, но специфичное воздействие лекарства на многие мишени, позволяющее добиться синергетического эффекта и более эффективного лечения комплексных заболеваний, таких как рак\cite{Anighoro2014}. 
	
	При этом полифармакология может выгодно отличаться от комбинирования нескольких лекарств, так как: (а) единственная молекула обычно имеет более предсказумую и безопасную фармакокинетику; (б) часто действующие на несколько мишеней лекарства имеют б\textbf{о}льшую эффективность на поздних стадиях заболевания; (в) не нужно учитывать эффекты перекрестного взаимодействия лекарств, которые, являясь негативными, переносятся хуже в случае комбинационной терапии; (г) при прочих равных меньше вероятность вырабатывания лекарственной устойчивости к одному лекарству, чем к хотя бы одному из набора лекарств \cite{Anighoro2014}.
	
	Стоит заметить, что каждый белковый домен в среднем содержит 3-5 связывающих карманов достаточного размера для связывания с типичными малыми лигандами \cite{SKOLNICK20151163}. Таким образом, существует возможность выбрать новый карман, отличный от ранее использовавшихся, для разработки лекарства. К тому же, количество видов связывающих карманов со статистически значимыми различиями оценивается, как меньшее 400 \cite{SKOLNICK20151163}, что позволяет считать полифармакологическую картину взаимодействий лиганд-мишень неизбежной, и потому более перспективной, чем таргетированная.
		
	Новая парадигма подчеркивает важность поиска всевозможных пар взаимодействий лиганд-мишень. Такой анализ может аккумулировать результаты уже известных связей, приводя к построению сложных сетей \cite{Anighoro2014}, но важнее уметь предсказывать такие взаимодействия. Так как перебор и оценка силы всех взаимодействий лиганд-мишень \textit{in vivo} и \textit{in vitro} является непрактичной, в этом направлениии развиваются компьютерные методы \cite{Chaudhari}.
	
	\section{Off-target взаимодействия}
	\textbf{Off-target} взаимодействия -- дополнительные взаимодействия выбранного лиганда/мишени с другими, кроме основных,  мишенями/лигандами. Одной из основных проблем в поиске таких взаимодействий исходя из структуры является то, что часто эти взаимодействия в большой степени определяются подвижными частями рецептора, которые сложно или невозможно исследовать с достаточной атомарной точностью \cite{Loving}.
	
	Принципиально, поиск субъектов \textbf{off-target} взаимодействия может осуществляться по структуре: (а) мишени; (б) лиганда; (в) связывающего сайта \cite{Rognan2010}.
	
	(а) при поиске возможных лигандов по известной структуре мишени, воспроизводится обычный процесс современного дизайна лекарств, при котором производится высокопроизводительный скрининг по базе возможных лигандов. Таким образом, в сущности, оценивается \textbf{druggability} рецептора. Процесс можно ускорить, используя поиск по так называемым <<горячим точкам>>, то есть набору мест на поверхности мишени, где максимальна энергия связывания с потенциальным лигандом \cite{Hall2015}, что напоминает концепцию фармакофорного поиска.
	 
	(б) поиск по структуре лиганда по сути своей близок к понятию \textbf{repurposing'a}, которое заключается в поиске новых мишеней и применений для лекарств, которые уже выпущены на рынок. Это позволяет сократить расходы на преклиническую стадию и оптимизацию \cite{March-Vila2017}. 
	\cite{Hall2015}
	
	
	(в) связывающие сайты могут сравниваться по различным характеристикам, таким как геометрические и физикохимические свойства поверхности мишени, профили взаимодействия или с структура остова. Также нахождение связывающих карманов само по себе сложная задача, к которой существует несколько подходов (\text{добавить, как ее решать})\cite{Ehrt2016}. 
	
	Связывающие сайты могут описываться разными способами. Например, как трехмерный граф из вершин-атомов, содиненный ребрами-длинами. Или же как облако точек, то есть чисто геометрически. В этих моделях могут выделяться основанные на фармакофорном принципе черты, которые в дальнейшем позволяют значительно ускорить поиск. Один из наиболее затратных в вычислениях, но и чувствительных методов -- построение карт электронной плотности\cite{Ehrt2016}.
	\cite{Cabrera}
	\cite{Brylinski}
	\cite{Govindaraj2018}
	\chapter{Методы}\
	%Протокол \cite{Tautermann2017}
	
	\textit{посмотреть МОЕ by Chemical Computing Group?}
	\section{Построение модели белка}
	Итогом развития геномного секвенирования в последнее время стал резкий рост количества известных белковых последовательностей, в то время, как только около одной сотой доли последовательностей охарактеризована с атомистической точностью и с использованием экспериментальных методов определения структуры \cite{Webb2017}.
	
	В таких условиях полученные компьютерными методами модели структур белков часто являются ценными для выдвижения проверяемых гипотез. Такие модели, в целом, создаются с использованием методов сравнительного моделирования или свободного моделирования (\textbf{free modelling}), также называемых <<ab initio>> или <<de novo>>\cite{Webb2017}.
	
	Сравнительное моделирование, или моделирование на основе гомологии (МГ), базируется на построении модели по известным структурам близких (\textbf{related} белков, как по шаблонам. Принцип \textbf{Free modelling} подхода в использовании не структуры близких белков, а в применении разнообразных методов, комбинирующих физику и известные особенности структур белков, например, сопоставление большого количества небольших фрагментов, выделенных из известных структур белков. Конструирование белков этим методом обычно чрезвычайно затратно в плане вычислений\cite{Webb2017}. Современные пакеты моделирования зачастую комбинируют эти два подхода, используя, если доступны шаблоны, МГ для построения основы-скелета белка, а затем уточняя положения петель, боковых цепей и частей без шаблона.
	
	Моделирование требует наличия схемы сэмплирования конформаций с целью получения набора альтернативных структур. Также необходима оценивающая функция для ранжирования этих конформаций по качеству. Для этих целей было предложено большое количество физически обоснованных (\textbf{physics-based}) функций энергии и статистических потенциалов, полученных из анализа известных структур \cite{BetterDOPE}.
	
	Так, например, MODELLER сначала накладывает входную последовательность на шаблонный остов, а затем, внося случайные смещения атомов, ищет локальный минимум оценивающей функции, повторяя эту процедуру несколько раз \cite{Webb2017}.  
	
	

	Значительное увеличение количества решенных структур GPCR позволяет строить с атомистической точностью пространственные структуры большого количества рецепторов при отсутствии экспериментально полученной структуры целевого белка \cite{Tautermann2017}.
	
	Построение модели состоит из нескольких этапов, результатом которых является физически и биологически адекватная модель белка. Ими являются: (1) определение сходства набора последотельностей с целевой и выделение шаблонов; (2) выравнивание целевой последовательности и шаблона(ов); (3) построение модели, основанной на выравнивании с выбранными шаблонами; (4) предсказание точности модели \cite{Webb2017}.
	\subsubsection{Выделение шаблонов и множественное выравнивание последовательностей}
	Сначала, беря аминокислотную последовательность заданного белка с неизвестной пространственной структурой, необходимо получить набор белков с известной 3D-структурой для последующего <<сшивания>> соответствующих участков.
	
	
	Для этого используются различные подходы к поиску шаблонов и множественному выравниванию соответствующих им последовательностей:
	
	1) PSI-BLAST\cite{ALTSCHUL1990403, Altschul1997}: с помощью матрицы вероятностей замен аминокислот BLOSUM62 и штрафа за пропуски производится поиск наиболее близких последовательностей в базе до некоторого порога близости. Далее, c использованием программы Clustal \cite{Thompson} эти последовательности выравниваются с исходной в совокупности. А именно, ... \textbf{алгоритм}
	
	2) HHBlits \cite{Remmert2011} -- быстрее и чувствительнее BLAST, но ... (\textbf{недостатки}). Использует предварительно кластеризованные с помощью kClustal \cite{Hauser2013} базы Uniprot и nr. \\
	Общее описание алгоритма: (а) приведение запроса из одной последовательности (выравнивание) или нескольких (множественное выравнивание) к скрытой марковской модели (СММ, сопоставление каждой позиции в последовательности вектора вероятностей размерности 1х20 обнаружить ту или иную аминокислоту на этом месте) производится добавлением к исходной  последовательностей, отличающихся от нее заменами аминокислот на похожие по физико-химическим свойствам. При этом учитывается локальный контекст в 13 оснований вокруг замены, а суммарная энергия этих замен должна быть меньше некоторого задаваемого порога, что важно для скорости и чувствительности алгоритма. Далее, каждая строка в исходной СММ причисляется к одному из 219 типичных профилей-кластеров последовательностей, создавая строку-профиль этой СММ. После этого сначала проводится выравнивание этого профиля СММ с предварительно созданными профилями из базы данных, потом -- обычное выравнивание уже входной последовательности внутри cоответствующего профилю СММ кластера базы данных. (\textbf{наверное, так много тут не надо, просто разбирался в алгоритме})
	
	3) GPCRdb.org -- при исследовании GPCR проще всего использовать готовые выравнивания с сайта базы данных GPCRdb (\textit{зачем тогда вообще другие нам?}).
	\subsubsection{Моделирование белка и оценка качества модели}
	
	Моделирование из выровненного набора шаблонов может производиться при помощи различных программных пакетов для предсказания четвертичной структуры белков. Среди них наиболее известным является MODELLER, но и, например, Rosetta \cite{ROHL200466} \cite{ Song2013}, Itasser\cite{iTasser}, RaptorX\cite{Raptor} могут быть использованы.
	\cite{Deng}
	
	При использовании пакета MODELLER в случае идентичности последовательностей более $30 \%$, в среднем более $\sim 60$ \% скелетных атомов моделируются корректно со средним квадратическим отклонением позиций $C \alpha$ атомов менее $3,5$ \AA. При меньшей идентичности, как правило, результат хуже \cite{Webb2017}.

	\textit{какие у разных программ особенности и различия?}
	
	Оценка качества модели \cite{DOPEShen2006}, 
	\subsubsection{Уточнение модели}
	Хотя современные программы МГ производят проверки физической и химической адекватности полученных моделей, за счет неточностей в выбранном силовом поле и просто недостаточности исходного покрытия последовательностями шаблонами, итоговые модели могут иметь значительные отличия от реальной структуры. 
	
	Улучшение точности модели может производиться средствами молекулярной динамики (МД) \cite{Nowroozi}. Производя множественные траектории МД с полученной МГ моделью, помещенной в нативную среду, могут быть получены несколько устойчивых подсостояний белка с локально минимальными свободными энергиями%\cite{TAUTERMANN2015111}
	, которые и будут считаться наиболее вероятными конформациями белка в мембране, что важно для GPCR.
	
	\section{Поиск off-target взаимодействий}
	Определение, где находится активный центр и взаимодействует ли лиганд с рецептором, может быть проведено различными методами. Возможно использование докинга
	\cite{Pradeep2012} и основанного на машинном обучении поиска....(\textit{что еще?})
	\\
	
	Заметки:\\
	слишком много слов <<взаимодействие>>, надо придумать синонимов 
	
	\bibliography{Diplom}
\end{document}